% Classe de Documento
\documentclass[12pt]{article}

% Packages de línguas

\usepackage[brazilian]{babel}
\usepackage[utf8]{inputenc}
\usepackage[T1]{fontenc}

% Packages de fontes
\usepackage{amsmath}
\usepackage{amssymb}
\usepackage{amsfonts}

% Packages de Imagens
\usepackage{graphicx}
\usepackage{caption}
\usepackage{subcaption}
\usepackage{float}
\graphicspath{ {logo/} }

% Packages de Headers e Footers   
%\usepackage{fancyhdr}
%\pagestyle{fancy}

% Packages de Array para Tabelas
\usepackage{array}
\usepackage{booktabs}
\usepackage{longtable}

%\lhead{Cabeçalho esquerdo} 
%\rhead{Cabeçalho direito}

% Packages adicionais
\usepackage{indentfirst} %identa o primeiro paragrafo de uma seção
\usepackage[top=1.5cm, bottom=2cm, right=2cm, left=2cm]{geometry} %redefine as margens

\begin{document}

    %título, grupo e participantes  
    \hfill \includegraphics[scale=0.1]{unicamp_logo} %inclui logo da Unicamp
    
    \begin{center}
        \textbf{\Large
            Relatório F 329 - Ponte de Wheatstone e Termistor
        }%título do relatório
        
        \textit{Experimento 2, grupo 6}
    
        %integrantes
        Gabriel Giacomini Marques, R.A: 172133 \\
        Giuseppe Tinti Tomio, R.A: 173511 \\
        Leonardo Martins Bianco, R.A: 178542 \\
        Letícia Fernandes Soriani, R.A: 178811
    \end{center}
    
    
    %afirmação de honestidade
    \def\abstractname{Afirmação de Honestidade} %Mudar o rótulo do abstract para que ele contenha a afirmação de honestidade
    \begin{abstract}
        A equipe declara que este relatório que está sendo entregue foi escrito por ela e que os resultados apresentados foram medidos por ela durante as aulas de F 329 no 1ºS/2017. Declara ainda que o relatório contém um texto original que não foi submetido anteriormente em nenhuma disciplina dentro ou fora da Unicamp.
    \end{abstract}

    
    %resumo
    \subsection*{Resumo}
        
    
    %objetivos
    \subsection*{Objetivos}
       
    Determinação da resistência R$_x$ de um resistor que equilibrava a ponte de Wheatstone e dos valores dos coeficientes A e B na equação típica do termistor.
    
    %metodologia	
    \subsection*{Metodologia}    
    
    A realização do experimento se dividiu em duas partes, ambas envolvendo a Ponte de Wheatstone (Fig. \ref{Rx} e \ref{Termistor}). Para sua realização, foram necessários um multímetro, uma fonte de energia, três resistências de valor nominal 100 $\Omega$, uma resistência de valor nominal 68 $\Omega$, uma resistência de década (variável), uma placa de acrílico para montagem dos circuitos, um termômetro de álcool e um termistor.
    
    % Quando colocar as imagens dos circuitos, olhar qual referência já está feita a ela no parágrafo anterior e colocá-las lado a lado (se necessário, checar código no roteiro do experimento 1)
    
    A parte I consistiu no cálculo da resistência R$_x$ de uma resistor de valor nominal de 68 $\Omega$ a partir do uso da resistência de década R$_d$. Sabendo os valores de R$_1$ e R$_2$ (medidos com o ohmímetro) e montando a ponte como indicado na Fig \ref{Rx}, foi possível calcular R$_x$ através da equação$^1$:    
    \begin{align}
        R_1 \cdot R_d = R_2 \cdot R_x
        \label{Encontrando Rx}
    \end{align}
    
    Nesta parte, R$_d$ foi ajustada para que a ponte estivesse em equlíbrio, ou seja, para que a voltagem V$_v$ indicada no voltímetro fosse de 0 V, ou próxima dela. Assim, calculou-se o valor de R$_x$ encontrando o valor de R$_d$ que fazia com que V$_v$ \approx 0V, 
    
    Já a parte II consistiu na determinação da equação característica de um termistor. Montando a ponte de Wheatstone como indicado na Fig. \ref{Termistor} e deixando uma das pontas do termistor mergulhada em água cujas temperaturas inicial se aproximava de sua temperatura de ebulição (100$^{\circ}$C) e final se aproximava da temperatura ambiente 
    (%VALOR DA TEMPERATURA AMBIENTE NO DIA DO EXPERIMENTO)
    ), ajustava-se, para cada temperatura, o valor de R$_d$ que fazia com que V$_v$ \approx 0V.
    
    
    %resultados obtidos
    \subsection*{Resultados}
    
        
    
    
    %análise de dados 
    \subsection*{Análise de Dados}
        
        
    
    
    %discussão
    \subsection*{Discussão}
        
       
    
    
    %conclusão
    \subsection*{Conclusão}
        
       
    
    
    %referências
    \subsection*{Referências}
    [1] Site efunda:
    (http://www.efunda.com/designstandards/sensors/methods/wheatstone_bridge.cfm)
    % O underline não está aparecendo no PDF, socorro!
           
\end{document}

